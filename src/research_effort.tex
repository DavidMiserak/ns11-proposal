\section{Research Effort}
Assigned: Carl Brandon, Peter Chapin, Hamid Ossareh

In a CubeSat flying formation, one of the most significant challenges
is maintaining proper attitude to ensure correct orbital trajectories
and thus proper functioning of the swarm. Traditionally, attitude
control is carried out by reaction wheels or magnetorquer
rods. Reaction wheels can be accurate but volume expensive and often
fail over time due to the constant rotation and high
torques. Magnetorquer rods do not require moving parts but provide
less torque and still take up substantial space. One of the key
aspects of this project is to design, test and fly solar panel
PCB-integrated magnetorquers which will be manufactured by partner
company LED Dynamics who provided solar panels for the VT Lunar
CubeSat mission. By integrating magnetorquers directly into the solar
panels, their volume will be effectively zero while still providing
the necessary torque for attitude control. The research will be based
upon existing literature which describes in detail the process of
optimizing geometry (i.e. trace width, number of traces) given initial
electric inputs (e.g. constant current, constant voltage)[citations here]. The general process will be to design a magnetorquer geometry based on the mission requirements and prior art, then submit that design to LED Dynamics for fabrication. The research will continue with the launch of the CubeSats where the efficacy of the magnetorquers in maintaining attitude throughout the formation change will be analyzed and corrected by the autonomous control algorithms.
