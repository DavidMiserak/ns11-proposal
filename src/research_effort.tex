\section{Research Effort}
Assigned: Carl Brandon, Peter Chapin, Hamid Ossareh

In a CubeSat flying formation, one of the most significant challenges
is maintaining proper attitude to ensure correct orbital trajectories
and thus proper functioning of the swarm. Traditionally, attitude
control is carried out by reaction wheels or magnetorquer
rods.

Reaction wheels can be accurate but volume intensive and often
fail over time due to the constant rotation and high
torques. Magnetorquer rods do not require moving parts but provide
less torque and still take up substantial space. One of the key
aspects of this project is to design, test and fly solar panel
PCB-integrated magnetorquers which will be manufactured by partner
company LED Dynamics who provided solar panels for the VT Lunar
CubeSat mission.

By integrating magnetorquers directly into the solar
panels, their volume will be effectively zero while still providing
the necessary torque for attitude control. The research will be based
upon existing literature which describes in detail the process of
optimizing geometry (i.e. trace width, number of traces) given initial
electric inputs (e.g. constant current, constant voltage)[cite_Source)

This path will open up the use of combination torquer-hysteresis for attitude adjustment and control for future CubeSat missions, as a means of direction control in LEO.

  numbers here]. The general process will be to design a magnetorquer geometry based on the mission requirements and prior art, then submit that design to LED Dynamics for fabrication. The research will continue with the launch of the CubeSats where the efficacy of the magnetorquers in maintaining attitude throughout the formation change will be analyzed and corrected by the autonomous control algorithms. The control system will be integrated into a board that will house the IMU and Lithium radio.

This research benefits the future field of large-scale construction in orbit by providing a base scaffold for maneuvering multiple swarm drones. It is conceivable that these  future drones can be made to work in a 3-dimensional or multi-layered 2-dimensional region of space, using radio, or light signaling from a stationary drone, or set of drones. This kind of work has been done on earth, using 2-D orbital construction protocols.(cite_bib)

Apart from construction in orbit and beyond is the acquisition of resources in space, namely asteroids. Given that asteroids are good sources of volatiles, propellants, construction materials, and precious metals (Cite_Source), there is a demand to make the acquisition process affordable, and repeatable.

Some of the current literature insists that the Mass Payback Ratio, startup costs, and risk associated with the current state of the art ensures the failure of any venture (Cite_Source). However, the article cited also notes that the following: "We find that from a profitability perspective, the throughput rate
and using smaller but multiple spacecraft per mission are key technical
parameters for reaching breakeven quickly" (Cite _Source).

Thus, the introduction of a few 1U cubesat drones, with swarm behavior could conceivably lower the cost and risk of capture and transportation of Near Earth Objects (NEOs), by providing relatively cheap artificial workers to retrieve and transport objects of interest to potential lagrange points, or resource centers in orbit for refinement.

Both large scale construction, and NEO capture could concievably make up the backbone of a linear refinement and production industry, where space production is preferable to terra ferma production.



