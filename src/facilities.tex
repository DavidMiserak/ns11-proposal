\section{Facilities}
Assigned: Avery Clotfelter

The primary facilities used for the development and control of the
cubesats are located at Vermont Technical College.  The college
already has an established ground station that was successfully used
to communicate with the [NAME] CubeSat.  The existing ground station
includes a two meter, and 70 cm high gain antennas, 1.2GHz receiver,
and 50 ft. Rohn 45G tower.  Also located at VTC are labs and
construction space for cubesat fabrication, including a new $12
million additive manufacturing lab with the potential to 3D print
electrospray thrusters.  A local small satellite propulsion company,
Benchmark Space Systems who is providing the propulsion systems is
also offering up its thermal vacuum chamber for environmental
testing.  At the University of Vermont, facilities include laboratory
and fabrication spaces, computing infrastructure with access to Matlab
and AGI STK, and business systems such as accounting, purchasing, and
a patent office.  The University of Vermont also has meeting areas
easily accessible by VTC students to support collaboration between
institutions.  Due to the previously established ground station and
CubeSat laboratory at VTC, UVM’s established infrastructure, and
Benchmark’s testing equipment, few expenses, if any, are expected to
be incurred in regards to facilities.
