\section{Abstract}
Assigned: David Demasi \cite{brandon:2008}

\begin{description}
\item[Principal Investigator] Carl Brandon
\item[University] Vermont Technical College
\end{description}

The research objectives of mission are as follows: To demonstrate
the developed CubedOS system by Vermont Technical College, written in
Spark Ada to remove runtime errors and decrease the chances of
software related mission failure, and demonstrate the maneuvering
capability of satellite swarms in LEO through the development of
algorithms by the University of Vermont.

Over the next two years, Both VTC and UVM will be developing these
overarching goals. This mission will seek to demonstrate the Cubed OS
system, running on SPARK Ada. This will all be done In tandem with an
additional technology demonstration of a cubesat swarm totaling three
1U CubeSats in LEO using a combination of magnetic torquers and
hysteresis rods for attitude control via adjustment and breaking
respectively. The control system algorithms for the swarm will be
developed by the University of Vermont in Simulink, and integrated
into CubedOS for the demonstration.

This inter-university collaboration effort is to be broken down as
follows: UVM will be developing the Algorithms for the swarm, VTC will
be using their software platform to develop the program, and the
assembly of the hardware will be developed jointly.

The intention is to launch the three 1U CubeSats into orbit, and use
the VTC ground station, used in a previous successful CubeSat mission,
to communicate with the swarm, and to update configuration orders if
need be. This is also combined with the interest in researching
non-toxic thruster systems for maneuvering systems via Benchmark Space
Systems.
	
The anticipated outcome is to build a foundation for any future
missions using CubedOS for single-sat or swarm-sat configurations and
to further integrate autonomous systems in the space industry through
this swarm demonstration, as well to increase inter-university
cooperation and student interest both in CubeSat hardware and software
development, while also paving the way for future university missions.
	
The impacts of this research are wide reaching, allowing for greater
autonomy in fleets of potential probes, and other deep space machines,
which require the ability to make complex decisions, such as collision
avoidance, without human intervention, as communication increases as a
function of irreducible distance. This will also provide a modular
software platform for other CubeSat missions to build their own
software from, whether university, DoD, NASA, or privately based.
